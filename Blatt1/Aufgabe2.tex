\section{Sonnenstrahlung}
\subsection{Teil a)}
Das Stefan-Boltzmann-Gesetz lautet:
\begin{center}
    $P=\sigma A T^4_{Sonne}$\\
\end{center}
mit:
\begin{center}
    $\sigma \approx 5.670 \times 10^{-8}\si[]{\frac{W}{m^2*K^4}}$\\
    $A=4\pi R^2_{Sonne}$\\
    $R_{Sonne}=6.96\times 10^8\si[]{m}$\\
    $T_{Sonne}=5780\si[]{K}$\\
    $\Rightarrow$ $P=5.670 \times 10^{-8}\si[]{\frac{W}{m^2*K^4}} *4\pi*(6.96\times 10^{8}\si[]{m})^2 *5780^4\si[]{K}=3.59\times 10^{26}\si[]{W}$

\end{center}
Das ist die komplette Strahlungsleistung der Sonne, im Abstand von einer Astronomischen Einheit (AE bzw. AU) 
verteilt sich diese Leistung auf die Oberfläche einer Kugel mit Radius $R=1\si[]{AE}\approx 150\times 10^9\si[]{m}$.
Es folgt also:
\begin{center}
    $\Phi=\frac{P}{A}$\\
    $A=4\pi R^2=4\pi*(150\times 10^{9}\si[]{m})^2=2.83\times 10^{23}\si[]{m^2}$\\
    $\Rightarrow$ $\Phi=1.272\si[]{\frac{kW}{m^2}}$
\end{center}
\subsection{Teil b)}
Die Strahlungsleistung welche die Erde trifft ergibt sich über:
\begin{center}
    $P_{Erde}=\Phi A_{Erde}$\\
    $A_{Erde}=\pi R^2_{Erde}$\\
    $\Rightarrow$ $P_{Erde}=\Phi \pi R^2_{Erde}$
\end{center}
Da allerdings 30\% der Strahlung sofort reflektiert werden muss noch ein Reflektionsfaktor hinzugefügt werden:
\begin{center}
    $P_{Erde}=\Phi \pi R^2_{Erde}(1-Albedo)$\\
    $\Rightarrow$ $P_{Erde}=1.272\si[]{\frac{kW}{m^2}}\pi *(6.36\times 10^6\si[]{m})^2*(1-0.3)=1.131\times 10^{17}\si[]{W}$
\end{center}

\subsection{Teil c)}
Die von der Erde absorbirte Strahlung lässt sich schreiben als:
\begin{center}
    $P_{abs}=(1-Albedo)\Phi A_{Erde1/2}=(1-Albedo)\frac{R^2_{Sonne}T^4_{Sonne}\sigma}{(1au)^2}\pi R^2_{Erde}$
\end{center}
Die von der Erde auf ihrer ganzen Fläche emmitierte Strahlung wird mit Stefan-Boltzmann geschrieben als:
\begin{center}
    $P_{Erde}=\sigma AT^4_{Erde}=4\sigma\pi R^2_{Erde}T^4_{Erde}$
\end{center}
Es muss nun gelten:
\begin{center}
    $P_{abs}=P_{Erde}$\\
    $\Rightarrow$ $(1-Albedo)\frac{R^2_{Sonne}T^4_{Sonne}\sigma}{(1au)^2}\pi R^2_{Erde}=4\sigma\pi R^2_{Erde}T^4_{Erde}$\\
    $\Leftrightarrow$ $(1-Albedo)\frac{R^2_{Sonne}T^4_{Sonne}}{(1au)^2}=4*T^4_{Erde}$\\
    $\Leftrightarrow$ $T^4_{Erde}=T^4_{Sonne}\frac{1-Albedo}{4}(\frac{R_{Sonne}}{1au})^2$\\
    $\Leftrightarrow$ $T_{Erde}=T_{Sonne} \sqrt[4]{\frac{1-Albedo}{4}(\frac{R_{Sonne}}{1au})^2}$\\
    $\Rightarrow$ $T_{Erde}=5780\si[]{K}*\sqrt[4]{\frac{1-0.3}{4}*(\frac{6.96\times 10^8\si[]{m}}{150 \times 10^9\si[]{m}})^2}=254.652\si[]{K}\approx 255\si[]{K}$
\end{center}

\subsection{Teil d)}
Es gibt zusätzlich das Problem das die Erde eine Atmosphäre hat und die Sonne nicht wie ein schwarzer Strahler nur im Infraroten Spektrum strahlt sondern auch noch mit einer
vielzahl anderen Wellenlängen von denen nicht alle durch die Atmosphäre abgeschirmt werden.